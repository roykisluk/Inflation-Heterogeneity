
\documentclass{beamer}
\title{Price Indexing Methodology and Analysis of the Consumer Expenditure Survey}
\author{Roy Kisluk}
\date{\today}

\begin{document}

\begin{frame}
    \titlepage
\end{frame}

\begin{frame}{Introduction}
    \begin{itemize}
        \item Our goal is to calculate consumption shares (weights) and price indexes for different demographics.
        \item First, we group households by different demographic characteristics.
        \item Second, we calculate the consumption weights for each group.
        \item Third, we calculate the Laspeyres index for each group.
        \item Fourth, we explore the results.
    \end{itemize}
\end{frame}

\begin{frame}{Grouping}
    \begin{itemize}
        \item We group the households by characteristics that are available in the data.
        \item Current groups include:
              \begin{itemize}
                  \item Nationality: Jewish, Arab, Other
                  \item Religious Observance: Secular, Conservative, Religious, Ultra-Orthodox, Mixed, Other
                  \item Age group of HHH: Young (18-25), Middle (26-64), Old (65+)
                  \item Income deciles or quintiles
                  \item Socioeconomic status of locality: quintiles or tertiles
                  \item Family size: No children, 1 to 3, 4 or more
              \end{itemize}
    \end{itemize}
\end{frame}


\begin{frame}{Household and Individual Data}
    \begin{itemize}
        \item Household-level data includes most of the demographic characteristics we need for grouping.
    \end{itemize}
    \begin{table}[h!]
        \centering
        \begin{tabular}{c c c c}
            \hline
            \textbf{misparmb} & \textbf{decile} & \textbf{\ldots} & \textbf{nationality} \\
            \hline
            57090             & 7               & \ldots          & Jewish               \\
            \hline
            57091             & 1               & \ldots          & Arab                 \\
            \hline
            57092             & 3               & \ldots          & Other                \\
            \hline
        \end{tabular}
        \label{tab:summary}
    \end{table}
    \begin{itemize}
        \item Individual-level data gives us the age.
    \end{itemize}
    \begin{table}[h]
        \centering
        \begin{tabular}{c c c c c c c}
            \hline
            S\_Seker & MisparMB & Prat & Weight  & Y\_Kirva & Y\_Kalkali & \ldots \\
            \hline
            2022     & 57089    & 1    & 411.796 & 1        & 1          &        \\
            2022     & 57090    & 1    & 213.598 & 1        & 1          &        \\
            2022     & 57090    & 2    & 213.598 & 2        & 2          &        \\
            \hline
        \end{tabular}
        \label{tab:sample_data}
    \end{table}
\end{frame}



\begin{frame}{Household and Individual Data}
    \begin{itemize}
        \item After grouping, we get a table with indicators per household ID (misparmb)
    \end{itemize}
    \begin{table}[h]
        \centering
        \begin{tabular}{c c c c c}
            \hline
            misparmb & Nationality & Age\_Group & \ldots & Family\_Size \\
            \hline
            57089    & Jewish      & Old        &        & no children  \\
            57090    & Jewish      & Old        &        & no children  \\
            57091    & Jewish      & Middle     &        & 1 to 3       \\
            57092    & Jewish      & Middle     &        & 1 to 3       \\
            57093    & Arab        & Middle     &        & no children  \\
            \hline
        \end{tabular}
        \label{tab:grouped_data}
    \end{table}
\end{frame}

\begin{frame}{Weights and Laspeyres Index}
    \begin{itemize}
        \item The goal now is to calculate the weights and Laspeyres index for each group:
    \end{itemize}
    $$
        I_{ij}=\frac{P_{ij}}{P_{oj}}
    $$
    $$
        W_{oj}=\frac{P_{oj}Q_{oj}}{\sum_{j\in L}P_{oj}Q_{oj}}
    $$
    $$
        I_{i}=\sum_{j\in L}W_{oj}I_{ij}\times 100
    $$
    \begin{itemize}
        \item $I_{i}$ - Index for the current period
        \item $Q_{oj}$ - Quantity of the good or service in the base period
        \item $P_{oj}$ - Price of the good or service in the base period
        \item $P_{ij}$ - Price of the good or service in the current period
        \item $L$ - The set of all goods and services in the index basket
    \end{itemize}
\end{frame}

\begin{frame}{Expenditure Data}
    \begin{itemize}
        \item Expenses data gives us the total expenditure on each product for each household. This is useful for calculating the weights. When looking at the base year: $Schum_{oj} = P_{oj}Q_{oj}$
    \end{itemize}
    \begin{table}[h]
        \centering
        \begin{tabular}{c c c c}
            \hline
            \textbf{misparmb} & \textbf{prodcode} & \textbf{schum} \\
            \hline
            57089             & 304170            & 5357.0         \\
            57089             & 304139            & 5160.0         \\
            57089             & 381012            & 723.0          \\
            57089             & 304014            & 4634.0         \\
            57089             & 304303            & 1259.0         \\
            \hline
        \end{tabular}
        \label{tab:expenses_table}
    \end{table}
\end{frame}

\begin{frame}{Survey Data}
    \begin{itemize}
        \item Survey data lets us estimate the price paid per unit of product. The variable $mehir$ represents the total expenditure on the product, as reported in the survey. $kamut$ represents the quantity of the product purchased.
        \item Using these variables, we can divide $mehir$ by $kamut$ to get the price paid per unit.
    \end{itemize}
    \begin{table}[h]
        \centering
        \begin{tabular}{c c c c}
            \hline
            \textbf{misparmb} & \textbf{prodcode} & \textbf{kamut} & \textbf{mehir} \\
            \hline
            57089             & 304170            & 1.0            & 18.0           \\
            57089             & 304139            & 2.0            & 4.0            \\
            57089             & 381012            & 1.0            & 44.0           \\
            57089             & 304014            & 1.0            & 6.0            \\
            57089             & 304303            & 18.0           & 18.0           \\
            \hline
        \end{tabular}
        \label{tab:survey_table}
    \end{table}
    \begin{itemize}
        \item We can then calculate the price index for each product $j$ by dividing the current price by the base price: $I_{ij}=\frac{P_{ij}}{P_{oj}}$
    \end{itemize}
\end{frame}

\begin{frame}{Weights and Prices}
    \begin{itemize}
        \item We get the following table, per year, per group:
    \end{itemize}
    \begin{tabular}{c c c c c}
        \hline
        \textbf{prodecode} & \textbf{weight} & \textbf{price} & \textbf{price\_base} & \textbf{price\_ratio} \\
        \hline
        300012             & 0.00024         & 6.21223        & 5.41153              & 1.14796               \\
        \hline
        300038             & 0.00055         & 7.54942        & 6.92654              & 1.08993               \\
        \hline
        300046             & 0.00173         & 13.10342       & 11.10066             & 1.18042               \\
        \hline
        300053             & 0.00087         & 9.57874        & 7.63743              & 1.25418               \\
        \hline
        300061             & 0.00151         & 6.25478        & 3.28928              & 1.90156               \\
        \hline
    \end{tabular}
\end{frame}

\begin{frame}{Laspeyres Index}
    \begin{itemize}
        \item We can then calculate the Laspeyres index for each group, by multiplying the weights by the price ratio and summing the results: $I_{i}=\sum_{j\in L}W_{oj}I_{ij}\times 100$
    \end{itemize}
    \begin{table}
        \begin{tabular}{c c c c c c}
            \hline
            \textbf{year} & \textbf{Secular} & \textbf{Conservative} & \textbf{Religious} & \textbf{Ultra-Orthodox} \\
            \hline
            2015          & 100.0            & 100.0                 & 100.0              & 100.0                   \\
            \hline
            2016          & 93.810           & 110.314               & 179.672            & 107.349                 \\
            \hline
            2017          & 98.875           & 102.746               & 110.247            & 102.271                 \\
            \hline
            2018          & 96.980           & 106.899               & 115.896            & 105.129                 \\
            \hline
            2019          & 102.574          & 108.910               & 121.337            & 103.093                 \\
            \hline
            2020          & 104.316          & 119.343               & 109.186            & 107.523                 \\
            \hline
            2021          & 107.229          & 115.962               & 107.235            & 114.162                 \\
            \hline
            2022          & 117.634          & 123.351               & 124.757            & 117.745                 \\
            \hline
        \end{tabular}
    \end{table}
    \begin{itemize}
        \item Further exploration and analysis of the data is shown separately.
    \end{itemize}
\end{frame}

\begin{frame}{Overview}
    \begin{itemize}
        \item In the following slides we take a look at the main variables available in the data
        \item The Consumer Expenditure Survey provides rich data for analyzing group-level inflation
        \item Contains detailed information on:
              \begin{itemize}
                  \item Household characteristics
                  \item Individual characteristics
                  \item Income sources
                  \item Expenditure patterns
              \end{itemize}
    \end{itemize}
\end{frame}

\begin{frame}{Key Household Variables}
    \begin{itemize}
        \item Number of individuals and providers in the household (HH)
        \item Nationality of the HH head (Jewish, Arab, Other)
        \item Locality, socioeconomic status, peripherality index
        \item Possession of durable goods (e.g.\ cars, computers) and access to services (e.g.\ internet, cable TV, central heating)
        \item Housing characteristics (e.g.\ ownership, number of rooms)
        \item Income and income sources (e.g.\ salary, self-employment, investments, benefits)
        \item Expenditure patterns (e.g.\ food, housing, transportation)
        \item Education type and level of the HH head
        \item Religion and religious observance level
    \end{itemize}
\end{frame}

\begin{frame}{Key Individual Variables}
    \begin{itemize}
        \item Age group (4-year intervals)
        \item Marital status and marriage year
        \item Immigrated from USSR
        \item Immigration year
        \item Continent of birth of each parent
        \item Education level, school type, last certification type, years of schooling
        \item Employment status, occupation, industry, work hours, work weeks
        \item Various disablity indicators
        \item Detailed income sources, including investments and benefits
    \end{itemize}
\end{frame}

\begin{frame}{Expenditure Data}
    \begin{itemize}
        \item Date of purchase, quantity, prices, estimated monthly expenditure per product
        \item Packaging type
        \item Retailer type
        \item Taxes and transfers
        \item Savings and investments
    \end{itemize}
\end{frame}

\begin{frame}{Expenditure Data}
    \begin{itemize}
        \item Food (bread and cereals, oils, meat and poultry, fish, dairy and eggs, sugar and related products, soft drinks, alcohol, meals outside home, fruits and vegetables)
        \item Housing expenses (water, electricity, gas, maintenance, housekeeping and cleaning, furniture, appliances, beddings and towels, local taxes, repairs, decorations)
        \item Clothing and footwear (clothing, footwear, cleaning, accessories)
        \item Health (medications, medical services, dental services, health insurance)
        \item Education and entertainment (education services, newspapers, books, cultural events, sports, hobbies, vacations, electronics)
        \item Transportation (public transportation, private transportation, fuel, maintenance, insurance, flights, mail and delivery, telecommunications)
        \item Other expenses (tobacco, cosmetics, law services, jewelry, baggage, charity)
    \end{itemize}
\end{frame}

\begin{frame}{Group Analysis Possibilities}
    Can analyze inflation differences by:
    \begin{itemize}
        \item Income levels (deciles)
        \item Socioeconomic status of locality
        \item Religion and religious observance level
        \item Education level
        \item Employment status
        \item Age group
    \end{itemize}
\end{frame}

\begin{frame}{Methodological Considerations}
    \begin{itemize}
        \item Sample weights available
        \item Detailed consumption categories
        \item Multiple income sources
        \item Rich demographic information
    \end{itemize}
\end{frame}

\begin{frame}{Limitations and Considerations}
    \begin{itemize}
        \item Cross-sectional nature of data
        \item Need to account for household composition
        \item Regional price variations
        \item Sampling methodology
        \item Response quality
    \end{itemize}
\end{frame}

\end{document}