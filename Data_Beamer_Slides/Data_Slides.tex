
\documentclass{beamer}
\title{Data Resources for Studying Group-Level Inflation Rates}
\author{Analysis of Israeli Household Expenditure Survey 2022}
\date{\today}

\begin{document}

\begin{frame}
    \titlepage
\end{frame}

\begin{frame}{Overview}
    \begin{itemize}
        \item The 2022 Household Expenditure Survey provides rich data for analyzing group-level inflation
        \item Contains detailed information on:
              \begin{itemize}
                  \item Household characteristics
                  \item Individual demographics
                  \item Income sources
                  \item Expenditure patterns
              \end{itemize}
    \end{itemize}
\end{frame}

\begin{frame}{Key Household Variables}
    \begin{itemize}
        \item Demographic Identifiers:
              \begin{itemize}
                  \item Household ID (misparMb)
                  \item Survey quarter (quarter)
                  \item Geographic location (yishuv, nafa)
              \end{itemize}
        \item Household Composition:
              \begin{itemize}
                  \item Number of persons (nefashot)
                  \item Number of standard persons (nefeshstandartit)
                  \item Number of income earners (mefarnasim)
              \end{itemize}
    \end{itemize}
\end{frame}

\begin{frame}{Socioeconomic Variables}
    \begin{itemize}
        \item Cultural/Religious Factors:
              \begin{itemize}
                  \item Nationality (nationality)
                  \item Religious observance level (RamatDatiyut)
                  \item Religion (religion)
              \end{itemize}
        \item Economic Status:
              \begin{itemize}
                  \item Socioeconomic cluster of residence (cluster)
                  \item Peripherality index (MadadPereferia)
              \end{itemize}
    \end{itemize}
\end{frame}

\begin{frame}{Expenditure Data}
    \begin{itemize}
        \item Total consumption expenditure (c3)
        \item Detailed categories:
              \begin{itemize}
                  \item Food (c30)
                  \item Housing (c32)
                  \item Household maintenance (c33)
                  \item Transportation and communication (c38)
                  \item Health (c36)
                  \item Education and leisure (c37)
              \end{itemize}
    \end{itemize}
\end{frame}

\begin{frame}{Income Data}
    \begin{itemize}
        \item Multiple income sources:
              \begin{itemize}
                  \item Gross monetary income (i1Kaspit)
                  \item Net monetary income (net)
                  \item Income from work (i11)
                  \item Benefits and transfers (i14)
              \end{itemize}
        \item Income classifications:
              \begin{itemize}
                  \item Salary income
                  \item Self-employment income
                  \item Pension and retirement income
              \end{itemize}
    \end{itemize}
\end{frame}

\begin{frame}{Group Analysis Possibilities}
    Can analyze inflation differences by:
    \begin{itemize}
        \item Geographic location
        \item Socioeconomic status
        \item Religious/cultural groups
        \item Household composition
        \item Income levels (deciles)
    \end{itemize}
\end{frame}

\begin{frame}{Methodological Considerations}
    \begin{itemize}
        \item Sample weights available (weight)
        \item Quarterly data collection
        \item Detailed consumption categories
        \item Multiple income sources
        \item Rich demographic information
    \end{itemize}
\end{frame}

\begin{frame}{Limitations and Considerations}
    \begin{itemize}
        \item Cross-sectional nature of data
        \item Need to account for household composition
        \item Regional price variations
        \item Sampling methodology
        \item Response quality
    \end{itemize}
\end{frame}

\end{document}